\documentclass[11pt,a4paper,oneside]{report}


\usepackage{amsmath,amssymb,calc,ifthen}

\usepackage{float}
%\usepackage{cancel}

\usepackage[table,usenames,dvipsnames]{xcolor} % for coloured cells in tables

\usepackage{tikz}

% Allows us to click on links and references!

\usepackage{hyperref}
\hypersetup{
    colorlinks,
    citecolor=black,
    filecolor=black,
    linkcolor=black,
    urlcolor=black
}

% Nice package for plotting graphs
% See excellent guide:
% http://www.tug.org/TUGboat/tb31-1/tb97wright-pgfplots.pdf
\usetikzlibrary{plotmarks}
\usepackage{amsmath,graphicx}
\usepackage{epstopdf}
\usepackage{caption}
\usepackage{subcaption}

% highlight - useful for TODOs and similar
\usepackage{color}
\newcommand{\hilight}[1]{\colorbox{yellow}{#1}}

\newcommand\ci{\perp\!\!\!\perp} % perpendicular sign
\newcommand*\rfrac[2]{{}^{#1}\!/_{#2}} % diagonal fraction

\usepackage{listings}



% margin size
\usepackage[margin=1in]{geometry}

\tikzstyle{state}=[circle,thick,draw=black, align=center, minimum size=2.1cm,
inner sep=0]
\tikzstyle{vertex}=[circle,thick,draw=black]
\tikzstyle{terminal}=[rectangle,thick,draw=black]
\tikzstyle{edge} = [draw,thick]
\tikzstyle{lo} = [edge,dotted]
\tikzstyle{hi} = [edge]
\tikzstyle{trans} = [edge,->]


\definecolor{mygreen}{rgb}{0,0.6,0}
\definecolor{mygray}{rgb}{0.5,0.5,0.5}
\definecolor{mymauve}{rgb}{0.58,0,0.82}

\lstset{ %
  backgroundcolor=\color{white},   % choose the background color; you must add 
%\usepackage{color} or \usepackage{xcolor}
  basicstyle=\footnotesize,        % the size of the fonts that are used for the 
%code
  breakatwhitespace=false,         % sets if automatic breaks should only happen 
%at whitespace
  breaklines=true,                 % sets automatic line breaking
  captionpos=b,                    % sets the caption-position to bottom
  commentstyle=\color{mygreen},    % comment style
  deletekeywords={...},            % if you want to delete keywords from the 
%given language
  escapeinside={\%*}{*)},          % if you want to add LaTeX within your code
  extendedchars=true,              % lets you use non-ASCII characters; for 
%8-bits encodings only, does not work with UTF-8
  frame=single,                    % adds a frame around the code
  keepspaces=true,                 % keeps spaces in text, useful for keeping 
%indentation of code (possibly needs columns=flexible)
  keywordstyle=\color{blue},       % keyword style
  language=Octave,                 % the language of the code
  morekeywords={*,...},            % if you want to add more keywords to the set
  numbers=left,                    % where to put the line-numbers; possible 
%values are (none, left, right)
  numbersep=5pt,                   % how far the line-numbers are from the code
  numberstyle=\tiny\color{mygray}, % the style that is used for the line-numbers
  rulecolor=\color{black},         % if not set, the frame-color may be changed 
%on line-breaks within not-black text (e.g. comments (green here))
  showspaces=false,                % show spaces everywhere adding particular 
%underscores; it overrides 'showstringspaces'
  showstringspaces=false,          % underline spaces within strings only
  showtabs=false,                  % show tabs within strings adding particular 
%underscores
  stepnumber=2,                    % the step between two line-numbers. If it's 
%1, each line will be numbered
  stringstyle=\color{mymauve},     % string literal style
  tabsize=2,                       % sets default tabsize to 2 spaces
  title=\lstname                   % show the filename of files included with 
%\lstinputlisting; also try caption instead of title
}


\title{Graphical Models Coursework 2 - Battleships problem}
\author{
    Razvan Valentin Marinescu\\
    \texttt{razvan.marinescu.14@ucl.ac.uk}
    \and
    Konstantinos Georgiadis\\
    \texttt{email@ucl.ac.uk}
}

\begin{document}
\belowdisplayskip=12pt plus 3pt minus 9pt
\belowdisplayshortskip=7pt plus 3pt minus 4pt

\maketitle{}

\section*{Contributions}

The problem was solved slightly different by both of us, but we get the same answer at the end. The algorithm and pseudo-code given below is Razvan's version.

\section*{Battleships Problem}

Without loss of generality, we can easily transform the given problem into the following equivalent problem. Find the pixel with the highest probability of containing a ship given the following constraints: 
\begin{itemize}
 \item ships cannot collide/overlap on the board
 \item one ship is oriented vertically, the other horizontally
 \item the ships cannot be placed on the blacklisted locations, given in the problem definition: [1, 10; 2, 2;3, 8; etc .. ]
\end{itemize}

The first two requirements were already given, while the third one was incorporated given the extra information about the 10 unsuccessful misses. We did a brute-force approach by trying to go through every possible “legal” position of each ships on the map. Whenever we managed to find a valid configuration, we incremented a counter for each of the occupied pixels. After all the possible configurations have been analysed and the counters of the respective pixels incremented, we normalise the counter matrix (divide each pixel by the sum of all the pixels in the matrix). However, there must be 10 pixels occupied on the board, so each pixel probability gets multiplied by 10. \textbf{The final result we get is that pixel at position (5,1) has the highest probability of being occupied, which is 0.2069}

For finding all the possible ways to position the boats on the board we do the following:
\begin{itemize}
 \item for each legal position of the first boat
  \begin{itemize}
  \item for each legal position of the second boat
      \begin{itemize}
      \item make sure boats don't collide
      \item increment counters on the grid of pixels.
      \end{itemize}
  \end{itemize}
\end{itemize}

The MATLAB code that implemented this brute-force approach is given below, with appropriate comments. 
\begin{lstlisting}
function prob120()

len = 10;
grid = zeros(len, len); % in each cell (i,j) contains how many possible scenarios exist where it is occupied by any of the boats

boatLen = 5;

illegal_locations = [1, 10; 2, 2;3, 8; 4, 4; 5, 6; 6, 5; 7, 4; 7, 7; 9, 2; 9, 9]';

% a few tests
testPoints = [ones(boatLen, 1) * 1, ones(boatLen, 1) * 6 + (0:boatLen-1)']'; 
assert(boat_illegal_location(testPoints, illegal_locations ) == 1);

testPoints = [ones(boatLen, 1) * 1, ones(boatLen, 1) * 5 + (0:boatLen-1)']'; 
assert(boat_illegal_location(testPoints, illegal_locations ) == 0);


% horizontal boat at position (iH,jH) occupies cells (iH, jH), (iH, jH+1), (iH,
% jH+2), (iH, jH+3) ...
% vertical boat at position (iV,jV) occupies cells (iV,jV), (iV+1,jV),
% (iV+2,jV) ...

% fix location of the horizontal ship
for iH=1:len
    for jH=1:(len-boatLen+1)
        % find cells the boat is occupying
        hBoatPoints = [ones(boatLen, 1) * iH, ones(boatLen, 1) * jH + (0:boatLen-1)']';  
        
        % make sure boat is not on illegal location
        if (boat_illegal_location(hBoatPoints, illegal_locations))
           continue; 
        end    
        
        % fix location of the vertical ship
        for iV=1:(len-boatLen+1)
            for jV=1:len
                % find cells the boat is occupying
                vBoatPoints = [ones(boatLen, 1) * iV  + (0:boatLen-1)', ones(boatLen, 1) * jV]'; 
                        
                % make sure boat is not on illegal location
                if (boat_illegal_location(vBoatPoints, illegal_locations))
                   continue; 
                end    
                
                % make sure boats don't collide
                if (boat_illegal_location(vBoatPoints, hBoatPoints))
                    %vBoatPoints
                    %hBoatPoints
                   continue; 
                end

                % update grid counters
                %[hBoatPoints, vBoatPoints]
                grid = update_grid(grid, [vBoatPoints, hBoatPoints]);
            end
        end
    end
end


grid = grid ./ sum(grid(:));

% multiply everything by 10 because there are 10 pixels that are actually
% occupied on the board (5 pixels for each ship)
grid = grid .* 10;
grid

max = 0;
maxI = 1; 
maxJ = 1;

for i=1:len
    for j=1:len
        if(grid(i,j) > max)
           maxI = i;
           maxJ = j;
           max = grid(i,j); 
        end
    end
end

max
[maxI, maxJ]

end

function is_on_illegal_loc = boat_illegal_location(boatPoints, illegal_locations)

is_on_illegal_loc = 0;

for point=boatPoints
    for loc=illegal_locations
        if (any(point - loc) == 0)
            is_on_illegal_loc = 1;
            return;
        end
    end
    
end

end

function grid = update_grid(grid, boatPoints)

for p=boatPoints
    grid(p(1),p(2)) = grid(p(1),p(2)) + 1;
end

end
\end{lstlisting}



\end{document}
